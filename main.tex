%--------------------------------------------------------

%--------------Preamble-------------------
\documentclass[letterpaper, 12pt]{article}

%% Language and font encodings
\usepackage[spanish]{babel}
\selectlanguage{spanish}
\usepackage[utf8x]{inputenc}
\usepackage[T1]{fontenc}

%% Sets page size and margins
\usepackage[letterpaper,top=3cm,bottom=3.5cm,left=2.8cm,right=2.8cm,marginparwidth=1.75cm]{geometry}

%% Useful packages
\usepackage{amsmath} % Algorithms
\usepackage{cprotect} % To use \verb as {argument}
\usepackage{indentfirst} % Verbose
\usepackage{graphicx}
\usepackage[colorinlistoftodos]{todonotes}
\usepackage[colorlinks=true, allcolors=blue]{hyperref}
\usepackage[procnames]{listings}
\usepackage{natbib}
\usepackage{url}
\usepackage{pdfpages} % For multpage pdf
\usepackage[linesnumbered,ruled]{algorithm2e}
\usepackage{algpseudocode}

%% MACROS
% Define keywords macro command
\providecommand{\keywords}[1]{\textbf{\textit{Keywords---}} #1}

\lstset{language=Python,
        basicstyle=\ttfamily\small,
        keywordstyle=\color{blue},
        commentstyle=\color{blue},
        stringstyle=\color{red},
        showstringspaces=false,
        identifierstyle=\color{purple},
        procnamekeys={def,class}}

%--------------/Preamble-------------------


%--------------Content-------------------
\title{Cryptosign\\
Firma de contratos electrónicos\\
}
%\author{}

\begin{document}
\maketitle

\begin{abstract}
Cryptosign es un servicio de gestión de documentos electrónicos altamente flexible. Tiene funciones de cifrado, almacenado, sellado, firmado y auditoría. Es compatible con diversos estándares y se puede utilizar con firmas autógrafas capturadas por medios electrónicos, certificados digitales (SSL, FIEL, etc) y contraseñas simples. Este documento explica las herramientas de firmado y auditoría y los beneficios comerciales.
\end{abstract}

% Insert keywords here
\setlength\parindent{.45in} \keywords{Blockchain}

\clearpage

\tableofcontents
\listoffigures
% \listoftables % Uncomment if necessary

%--------------Glossary-------------------
% \newpage % Uncomment if you want glossary on same page
% {\Large \textbf{Glossary of terms and names}}
\section*{Glosario}
\begin{enumerate}
    \item 
\end{enumerate}

%--------------/Glossary-------------------

\clearpage

% \subsection{How to add Comments}
% {\Large If you're contributing, label the document {\bf before} and {\bf after} your contribution (on top menu, {\bf ``History \& Revisions'' > type name and click ``Add Label''}) to help us keep track of all contributions.}\\

% Comments can be added to your project by clicking on the comment icon in the toolbar above. % * <john.hammersley@gmail.com> 2016-07-03T09:54:16.211Z:
% %
% % Here's an example comment!
% %
% To reply to a comment, simply click the reply button in the lower right corner of the comment, and you can close them when you're done.

% Comments can also be added to the margins of the compiled PDF using the todo command\todo{Name: Here's a comment in the margin!}, as shown in the example on the right. You can also add inline comments:\\ {\bf Please add your name so we can keep track of who's commenting.}

% \todo[inline, color=red!40]{Name: This is an inline comment.}




\section{Introducción}
Esto es una introducción
\subsection{Estado del Arte}
Esto es una subsección
\subsubsection{Sub}
Esto es otra subsección
\section{Legal}
\subsection{Framework}

	

\clearpage

\bibliographystyle{plainnat}
\bibliography{biblio}


\end{document}
%--------------/Content-------------------
